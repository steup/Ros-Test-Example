\documentclass{beamer}
\setbeamertemplate{navigation symbols}{}
\usetheme{Frankfurt}

\usepackage{listings}
\lstdefinestyle{code}{
  language       = C++,
  basicstyle      = \footnotesize\ttfamily\color{gray},
  numbers         = left,
  tabsize         = 2,
  backgroundcolor = \color{black},
  keywordstyle    = \bfseries\color{green},
  commentstyle    = \itshape\color{cyan},
  identifierstyle = \color{white},
  stringstyle     = \color{orange},
  numberstyle     = \color{red}
}
\lstdefinestyle{cmd}{
  basicstyle      = \scriptsize\ttfamily\color{green},
  backgroundcolor = \color{black}
}

\usepackage{hyperref}
\hypersetup{colorlinks=true,linkcolor=black,urlcolor=cyan}

\usepackage{etoolbox}

\begin{document}

\newcommand{\defEmph}[2]{%
	\expandafter\newcommand\csname #1\endcsname{%
		\href{#2}{\textit #1}%
	}%
  \newtoggle{#1}%
	\toggletrue{#1}%
}
\renewcommand{\emph}[1]{%
	\iftoggle{#1}{%
		\expandafter\csname #1\endcsname%
		\global\togglefalse{#1}%
	}{%
		\textit #1%
	}%
}

\defEmph{roslaunch}{http://wiki.ros.org/roslaunch}
\defEmph{rostest}{http://wiki.ros.org/rostest}
\defEmph{RVIZ}{http://wiki.ros.org/rviz}
\defEmph{catkin}{http://wiki.ros.org/catkin}
\defEmph{roscore}{http://wiki.ros.org/roscore}
\defEmph{GTest}{https://github.com/google/googletest}
\defEmph{git}{https://git-scm.com/}
\defEmph{GMock}{https://github.com/google/googlemock}
\defEmph{selftest}{http://wiki.ros.org/self\_test}
\defEmph{hztest}{http://wiki.ros.org/rostest/Nodes}

\title{Extended Testing using GMock and ROS::SelfTest}   
\author{Christoph Steup} 
\date{November 3, 2015} 

\frame{\titlepage} 

\frame{\frametitle{Table of contents}\tableofcontents} 

\section{Summary: Unit-Testing}
\frame{\frametitle{Interface: DefinitionSome Examples}
  \begin{itemize}
    \item Testing small units of Code
    \item Simple tests expressing the expected behaviour
    \item Automated test execution to catch regressions
    \item Integration of the tests to the build process (catkin, rostest)
  \end{itemize}
}

\section{Interfaces}
\frame{\frametitle{Interface: Definition}
  \begin{block}{Definition: Specification}{}
  \begin{block}{Definition: Interface}{}
}

\frame{\frametitle{Flavours of Interfaces}
 \begin{description}
  \item[Class Interface]
  \item[Library Interface]
  \item[Communication Interface]
  \item[User-Interface]
 \end{description}
}

\frame{\frametitle{Interface Hierarchy}

}

\frame{\frametitle{Benefits}
  \begin{block}{Decoupling}{}
  \begin{block}{Testing}{}
  \begin{block}{Mocking}{}
}

\begin{frame}[fragile]
\frametitle{Example: Decoupling}

\end{frame}

\begin{frame}[fragile]
\frametitle{Example: Testing}

\end{frame}

\begin{frame}[fragile]
\frametitle{Example: Mocking}

\end{frame}



\end{document}
